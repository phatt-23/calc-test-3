\documentclass[10pt]{article}

%%%%%%%%%%%%%%%%%%%%%
%%   The Preamble  %%
%%%%%%%%%%%%%%%%%%%%%

\usepackage{graphicx}
\usepackage{amsmath,amsfonts,theorem,amssymb,mathtools}
\usepackage{array, multirow}
\usepackage[svgnames]{xcolor}
\usepackage{pgfplots}
\usepackage{bm}                     % for bold math expressions
\usepackage{etoolbox}               % for the setcounter
\usepackage{xifthen}
\usepackage[
    tmargin=2cm,
    rmargin=1in,
    lmargin=1in,
    margin=0.85in,
    bmargin=2cm,
    footskip=0.4in
]{geometry}
\usepackage[utf8]{inputenc}
\usepackage{textcomp}
\usepackage{verbatim}
\usepackage[czech]{babel}
\usepackage{lmodern}
\usepackage{enumitem}
\usepackage[T1]{fontenc}
\usepackage{tikz}
\usepackage{fancyhdr}
\usepackage{lastpage}
\usepackage{atveryend}
\usepackage[super]{nth}
\usepackage[many]{tcolorbox}
\usepackage{pgfkeys}
\usepackage{lipsum}
\usepackage{multicol}
\usepackage{hyphenat}
\usepackage[outputdir=../]{minted}
\usepackage{fancyvrb}
\usepackage[czech=quotes]{csquotes}
\usepackage{float}
\usepackage{listings}
\usepackage{listing}
\usepackage{epigraph}
\usepackage{hyperref}
\usepackage{tikzpagenodes}
\usepackage[explicit]{titlesec}
\usepackage[font=small, labelfont=bf]{caption}
\usepackage[protrusion=true, expansion=true]{microtype}
\usepackage{sectsty}
\usepackage{wrapfig, setspace}

\fancypagestyle{mycustomstyle}{
    \fancyhf{}                                      % clear header and footer
    \fancyhead[L]{\nouppercase{\leftmark}}          % section name on the left
    \fancyhead[R]{\thepage}                         % page number on the right
    \renewcommand{\headrulewidth}{0.4pt}            % add a rule below the header
    \renewcommand{\footrulewidth}{0.4pt}            % add rule above the footer
    \fancyfoot[C]{\emph{\textcircled{phatt}}}       % centered text in the footer
    \fancyfoot[R]{Strana \thepage\ z \pageref{LastPage}} 
    % page number on the right side of the footer
}
\setlength{\headheight}{15pt}

\usetikzlibrary{arrows}
\pgfplotsset{compat=1.9, width=10cm}

\setlength{\tabcolsep}{18pt}                        % sets a wider white-space in the table box 
\renewcommand{\arraystretch}{1.5}                   % sets a taller white-space in the table box
\setlength{\arrayrulewidth}{0.3mm}                  % sets the thickness of the borders

\definecolor{gre}{RGB}{101, 191, 127}
\definecolor{gree}{RGB}{7, 135, 44}
\definecolor{listinggray}{gray}{0.9}
\definecolor{lbcolor}{rgb}{0.9,0.9,0.9}
\definecolor{Darkgreen}{rgb}{0,0.4,0}
\lstset{
    backgroundcolor=\color{lbcolor},
    tabsize=4,    
%   rulecolor=,
    language=C++,
        basicstyle=\scriptsize,
        upquote=true,
        aboveskip={1.5\baselineskip},
        columns=fixed,
        showstringspaces=false,
        extendedchars=false,
        breaklines=true,
        prebreak = \raisebox{0ex}[0ex][0ex]{\ensuremath{\hookleftarrow}},
        frame=single,
        numbers=left,
        showtabs=false,
        showspaces=false,
        showstringspaces=false,
        identifierstyle=\ttfamily,
        keywordstyle=\color[rgb]{0,0,1},
        commentstyle=\color{Darkgreen},
        stringstyle=\color[rgb]{0.627,0.126,0.941},
        numberstyle=\color[rgb]{0.205, 0.142, 0.73},
%        \lstdefinestyle{C++}{language=C++,style=numbers}’.
        captionpos=b,
}

\definecolor{myg}{RGB}{56, 140, 70}
\definecolor{myb}{RGB}{45, 111, 177}
\definecolor{myr}{RGB}{199, 68, 64}
\definecolor{notesgreen}{RGB}{0,162,0}
\definecolor{myp}{RGB}{197, 92, 212}
\definecolor{mygr}{HTML}{2C3338}
\definecolor{myred}{RGB}{127,0,0}

\hypersetup{
    colorlinks=true,
    linkcolor=blue,
    filecolor=magenta,      
    urlcolor=cyan,
    pdftitle={Overleaf Example},
    pdfpagemode=FullScreen,
}

\urlstyle{same}


%%%%%%%%%%%%
%% Macros %%
%%%%%%%%%%%%

%%%%%%%%%%%%%%%%%%%%%
%%   The Preamble  %%
%%%%%%%%%%%%%%%%%%%%%

\usepackage{graphicx}
\usepackage{amsmath,amsfonts,theorem,amssymb,mathtools}
\usepackage{array, multirow}
\usepackage[svgnames]{xcolor}
\usepackage{pgfplots}
\usepackage{bm}                     % for bold math expressions
\usepackage{etoolbox}               % for the setcounter
\usepackage{xifthen}
\usepackage[
    tmargin=2cm,
    rmargin=1in,
    lmargin=1in,
    margin=0.85in,
    bmargin=2cm,
    footskip=0.4in
]{geometry}
\usepackage[utf8]{inputenc}
\usepackage{textcomp}
\usepackage{verbatim}
\usepackage[czech]{babel}
\usepackage{lmodern}
\usepackage{enumitem}
\usepackage[T1]{fontenc}
\usepackage{tikz}
\usepackage{fancyhdr}
\usepackage{lastpage}
\usepackage{atveryend}
\usepackage[super]{nth}
\usepackage[many]{tcolorbox}
\usepackage{pgfkeys}
\usepackage{lipsum}
\usepackage{multicol}
\usepackage{hyphenat}
\usepackage[outputdir=../]{minted}
\usepackage{fancyvrb}
\usepackage[czech=quotes]{csquotes}
\usepackage{float}
\usepackage{listings}
\usepackage{listing}
\usepackage{epigraph}
\usepackage{hyperref}
\usepackage{tikzpagenodes}
\usepackage[explicit]{titlesec}
\usepackage[font=small, labelfont=bf]{caption}
\usepackage[protrusion=true, expansion=true]{microtype}
\usepackage{sectsty}
\usepackage{wrapfig, setspace}

\fancypagestyle{mycustomstyle}{
    \fancyhf{}                                      % clear header and footer
    \fancyhead[L]{\nouppercase{\leftmark}}          % section name on the left
    \fancyhead[R]{\thepage}                         % page number on the right
    \renewcommand{\headrulewidth}{0.4pt}            % add a rule below the header
    \renewcommand{\footrulewidth}{0.4pt}            % add rule above the footer
    \fancyfoot[C]{\emph{\textcircled{phatt}}}       % centered text in the footer
    \fancyfoot[R]{Strana \thepage\ z \pageref{LastPage}} 
    % page number on the right side of the footer
}
\setlength{\headheight}{15pt}

\usetikzlibrary{arrows}
\pgfplotsset{compat=1.9, width=10cm}

\setlength{\tabcolsep}{18pt}                        % sets a wider white-space in the table box 
\renewcommand{\arraystretch}{1.5}                   % sets a taller white-space in the table box
\setlength{\arrayrulewidth}{0.3mm}                  % sets the thickness of the borders

\definecolor{gre}{RGB}{101, 191, 127}
\definecolor{gree}{RGB}{7, 135, 44}
\definecolor{listinggray}{gray}{0.9}
\definecolor{lbcolor}{rgb}{0.9,0.9,0.9}
\definecolor{Darkgreen}{rgb}{0,0.4,0}
\lstset{
    backgroundcolor=\color{lbcolor},
    tabsize=4,    
%   rulecolor=,
    language=C++,
        basicstyle=\scriptsize,
        upquote=true,
        aboveskip={1.5\baselineskip},
        columns=fixed,
        showstringspaces=false,
        extendedchars=false,
        breaklines=true,
        prebreak = \raisebox{0ex}[0ex][0ex]{\ensuremath{\hookleftarrow}},
        frame=single,
        numbers=left,
        showtabs=false,
        showspaces=false,
        showstringspaces=false,
        identifierstyle=\ttfamily,
        keywordstyle=\color[rgb]{0,0,1},
        commentstyle=\color{Darkgreen},
        stringstyle=\color[rgb]{0.627,0.126,0.941},
        numberstyle=\color[rgb]{0.205, 0.142, 0.73},
%        \lstdefinestyle{C++}{language=C++,style=numbers}’.
        captionpos=b,
}

\definecolor{myg}{RGB}{56, 140, 70}
\definecolor{myb}{RGB}{45, 111, 177}
\definecolor{myr}{RGB}{199, 68, 64}
\definecolor{notesgreen}{RGB}{0,162,0}
\definecolor{myp}{RGB}{197, 92, 212}
\definecolor{mygr}{HTML}{2C3338}
\definecolor{myred}{RGB}{127,0,0}

\hypersetup{
    colorlinks=true,
    linkcolor=blue,
    filecolor=magenta,      
    urlcolor=cyan,
    pdftitle={Overleaf Example},
    pdfpagemode=FullScreen,
}

\urlstyle{same}



% Limit = use this in the $math$ mode 
\newcommand{\Lim}[1]{\raisebox{0.5ex}{\scalebox{0.8}{$\displaystyle \lim_{#1}\;$}}}
\newcommand*\diff{\mathop{}\!\mathrm{d}}            % \diff{x} -> dx
\newcommand*\Diff[1]{\mathop{}\!\mathrm{d^#1}}      % \Diff{3}{y} -> d^{3}y
\newcommand{\ts}{\textsuperscript}                  % shortcut for superscript in text

%% NEW COMMANDS %%%
\newcommand*\qs[2]{
	\begin{center}
    \begin{tcolorbox}[
        title=\textbf{#1},
        colback=gre!9!white,
        colframe=gre!40!black,
        colbacklower=gre!40!black,
        skin=bicolor,
        halign lower=flush right,
        collower=white,
		attach boxed title to top left={
			xshift=2mm,
			yshift=-2mm
		},
        boxed title style={
            size=small,
            colback=gre!57!black,
        },
        drop shadow=black!50!white,%
		width=0.7\textwidth,
        bottom=-3mm,
    ]
	    #2
        \tcblower
    \end{tcolorbox}
	\end{center}

}

\newcommand*\dfn[2]{
    \begin{center}
        \begin{tcolorbox}[
            title=\textbf{#1},
            % halign=center,
            % valign=center,
            nobeforeafter,
            colback=myred!5!white,
            colfafparticlerame=myred!75!black,
            colbacklower=myred!75!black,
	        colframe=myred!75!black,
            skin=bicolor,
            halign lower=flush right,
            bottom=-3mm,
            collower=white,
            % breakable,
    		attach boxed title to top left={
    			xshift=2mm,
    			yshift=-2mm
    		},
            boxed title style={
                size=small,
                colback=myred!90!black,
            },
			width=0.7\textwidth,
            drop shadow=black!50!white,%
        ] 
            #2 
            \tcblower
        \end{tcolorbox}
    \end{center}
}

\newcommand*\note[2]{
	\begin{tcolorbox}[
		enhanced,
        colback=myb!5!white,
        colframe=myb!50!black,
        colbacklower=myb!50!black,
		title=\textbf{#1},
		attach boxed title to bottom left={
			xshift=2mm,
			yshift=2mm
		},
		boxed title style={
			size=small,
			colback=myb!75!black
		},
		width=0.5\textwidth,
	    drop shadow=black!50!white,%
        bottom=-3mm,
	]
	    #2
        \tcblower
	\end{tcolorbox}
}

\newcommand*\mathblock[1]{
	\begin{align}
		#1
	\end{align}
}

\newcommand*{\mathblocks}[1]{
	\begin{align*}
		#1
	\end{align*}
}


\newcommand*{\ntoinf}{n \rightarrow +\infty}
\newcommand*{\xtoinf}{x \rightarrow +\infty}





\title{\Huge{\textbf{MA1 Integrály}} \\ \huge{}}
\author{Phat Tran}
\date{\today}

\begin{document}
	\maketitle
	\thispagestyle{empty}
	\setcounter{page}{0}
	\newpage
	\pagestyle{mycustomstyle} 

	\section{Integrály}
		\begin{align*}
			\int f(x) \diff{x} &\, \ldots \, \text{neurčitý integrál} \\
			\int f(x) \diff{x} = F(x) &\, \ldots \, \text{primitivní funkce k funkci $f(x)$} \\
			F'(x) = f(x) &\, \ldots \, \text{derivace primitivní funkce, $F(x)$, je funkce $f(x)$} \\
		\end{align*}

		\defi{Pravidlo sumy}{$$
			\int \Big( f(x) \pm g(x) \Big) \diff{x} = \int f(x) \diff{x} \pm \int g(x) \diff{x}
			$$}

		\defi{Pravidlo konstanty}{$$\int k f(x) \diff{x} = k \int f(x) \diff{x}$$}

		\defi{Metoda Per Partes}{
			\begin{align*}
            	\left( f \cdot g \right)' &= f'g + fg' \\
            	\int \left( f \cdot g \right)' &= \int f'g + \int fg' \\
            	f \cdot g &= \int f'g + \int fg' 
            \end{align*}
			\tcbline
        	$$ \textbf{I.} \, \int f'g = fg - \int fg' $$ 
        	$$ \textbf{II.} \, \int fg' = fg - \int f'g $$ 
		}

		\defi{Subtituční metoda}{
			\begin{flushleft}
				Pokud pro funkce $f$ a $g$ platí, že
				\begin{enumerate}
	            	\item funkce $f$ má primitivní funkci $F$ na intervalu $(a, b)$,
					\item funkce $g$ je na intervalu $(\alpha, \beta)$ diferenciovatelná
					\item a $g((\alpha, \beta)) \subset (a, b)$,
	            \end{enumerate}
				pak funkce $f\left(g(x)\right) \cdot g'(x)$ má primitivní funkci na intervalu $(\alpha, \beta)$ a platí
				$$ \int f\left(g(x)\right) \cdot g'(x) = F\left(g(x)\right). $$
			\end{flushleft}
			\tcbline
			$$\int f\left( g(x) \right) g'(x) \diff{x} = \int f(t) \diff{t} \, \text{, kde}$$
			$$t = g(x) \, \text{;} \, \diff{t} = g'(x) \diff{x}. $$
		}
		
		\pagebreak

	\section{Příklady}
		\qs{Vypočítej integrál}{ $$ \int \pi x^{2024} \diff{x} $$ }
			\begin{align}
				&= \int \pi x^{2024} \diff{x} \\ 
				&= \pi \int x^{2024} \diff{x} \\ 
				&= \pi \frac{x^{2025}}{2025} \\ 
				&= \frac{\pi}{2025} x^{2025} + C
			\end{align}

		\qs{Vypočítej integrál}{ $$ \int (3x - 2) e^x \diff{x} $$ }
			\note{Vybrání funkce $f(x)$ a $g(x)$.}{
				$$f(x) = 3x - 2 \, &\rightarrow \, f'(x) = 3$$
				$$g'(x) = e^x \, &\rightarrow \, g(x) = e^x$$
			}
			\begin{align}
            	&= (3x - 2) e^x - \int 3e^x \diff{x} \\
            	&= (3x - 2) e^x - 3e^x \\
            	&= e^x\left((3x - 2) - 3\right) \\
            	&= e^x\left(3x - 5\right) + C
            \end{align} 

		\qs{Vypočítej integrál}{ $$ \int (2x + 3) \cos{3x} \diff{x} $$ }
			\note{Vybrání funkce $f(x)$ a $g(x)$}{
				$$ f(x) = 2x + 3 \, \rightarrow \, f'(x) = 2 $$
				$$ g'(x) = \cos{3x} \, \rightarrow \, g(x) = -\frac{1}{3}\sin{3x} $$
			}
			\mathblock{
				&= -\frac{1}{3}\sin{3x}(2x + 3) + \frac{2}{3} \int \sin{3x} \diff{x} \\
				&= -\frac{1}{3}\sin{3x}(2x + 3) + \frac{2}{3} \left(-\frac{1}{3}\cos{3x}\right) \\
				&= -\frac{1}{3}\sin{3x}(2x + 3) - \frac{2}{9} \cos{3x} + C 
			}

		\qs{Vypočítej integrál}{ $$ \int (x^2 - x) e^{4x} \diff{x} $$ }
			\note{1. Vybrání funkce $f(x)$ a $g(x)$}{
				\mathblocks{
					f(x) = x^2 - x \, \rightarrow \, f'(x) = 2x - 1 \\
					g'(x) = e^{4x} \, \rightarrow \, g(x) = \frac{1}{4}e^{4x}
				}
			}
			\mathblock{
				&= \frac{1}{4}(x^2 - x)e^{4x} - \frac{1}{4} \int (2x - 1)e^{4x} \diff{x}
			}
			\note{2. Vybrání funkce $f(x)$ a $g(x)$}{
				\mathblocks{
					f(x) = 2x - 1 \, \rightarrow \, f'(x) = 2\\
					g'(x) = e^{4x} \, \rightarrow \, g(x) = \frac{1}{4}e^{4x}
				}
			}
			\mathblock{
				&= \frac{1}{4}(x^2 - x)e^{4x} - \frac{1}{4} \left( \frac{1}{4}(2x - 1)e^{4x} - \frac{1}{2} \int e^{4x} \diff{x} \right)\\
				&= \frac{1}{4}(x^2 - x)e^{4x} - \frac{1}{4} \left( \frac{1}{4}(2x - 1)e^{4x} - \frac{1}{8} e^{4x} \right)\\
				&= \frac{1}{4}(x^2 - x)e^{4x} - \frac{1}{16}(2x - 1)e^{4x} + \frac{1}{32} e^{4x} \\
				&= \frac{1}{32}e^{4x}\left( 8(x^2 - x) - 2(2x - 1) + 1 \right) \\
				&= \frac{1}{32}e^{4x}\left( 8x^2 - 8x - 4x - 2 + 1 \right) \\
				&= \frac{1}{32}e^{4x}\left( 8x^2 - 12x - 1 \right) + C
			}

		\qs{Vypočítej integrál}{ $$ \int \ln{x} \diff{x} $$ }
			\note{Vybrání funkce $f(x)$ a $g(x)$}{
				$$ f(x) = \ln{x} \, \rightarrow \, f'(x) = \frac{1}{x} $$
				$$ g'(x) = 1\, \rightarrow \, g(x) = x $$
			}
			\mathblock{
				&= x\ln{x} - \int \frac{1}{x} x \diff{x} \\
				&= x\ln{x} - \int 1 \diff{x} \\
				&= x\ln{x} - x \\
				&= x \left( \ln{x} - 1 \right) + C
			}

		\qs{Vypočítej integrál}{ $$ \int \ln^2{x} \diff{x} $$ }
			\note{Vybrání funkce $f(x)$ a $g(x)$}{
				$$ f(x) = \ln^2{x} \, \rightarrow \, f'(x) = 2\ln{x} \frac{1}{x} $$
				$$ g'(x) = 1 \, \rightarrow \, g(x) = x $$
			}
			\mathblock{
				x \ln^2{x} - \int 2\ln{x} \cdot \frac{1}{x} x \diff{x} \\
				x \ln^2{x} - 2 \int \ln{x} \cdot 1 \diff{x} \\
				x \ln^2{x} - 2 \int \ln{x} \diff{x} \\
				x \ln^2{x} - 2 \left( x \left( \ln{x} - 1 \right) \right) \\
				x \ln^2{x} - 2x\ln{x} + 2x \\
				x \left( \ln^2{x} - 2\ln{x} + 2 \right) + C
			}

		\qs{Vypočítej integrál}{ $$ \int x \arctan{x} \diff{x} $$ }
			\note{Vybrání funkce $f(x)$ a $g(x)$}{
				$$ f(x) = \arctan{x} \, \rightarrow \, f'(x) = \frac{1}{1 + x^2} $$
				$$ g'(x) = x \, \rightarrow \, g(x) = \frac{1}{2}x^2 $$
			}
			\mathblock{
				&= \frac{1}{2} x^2 \arctan{x} - \int \frac{1}{1 + x^2} \frac{1}{2}x^2  \diff{x} \\
				&= \frac{1}{2} x^2 \arctan{x} - \frac{1}{2}\int \frac{1}{1 + x^2} x^2  \diff{x} 
			}
			\note{Úprava výrazu}{
				$$ \frac{1}{1 + x^2} x^2 = \frac{x^2}{1 + x^2} = \frac{x^2 + 1 - 1}{x^2 + 1} =$$
				$$ = 1 - \frac{1}{x^2 + 1} $$
			}
			\mathblock{
				&= \frac{1}{2} x^2 \arctan{x} - \frac{1}{2} \left( \int  1 - \frac{1}{x^2 + 1}  \diff{x} \right) \\
				&= \frac{1}{2} x^2 \arctan{x} - \frac{1}{2} \left( \int 1 \diff{x} - \int \frac{1}{x^2 + 1}  \diff{x} \right) \\
				&= \frac{1}{2} x^2 \arctan{x} - \frac{1}{2} \left( x - \arctan{x} \right) \\
				&= \frac{1}{2} \left( x^2 \arctan{x} - \left( x - \arctan{x} \right) \right) \\
				&= \frac{1}{2} \left( x^2 \arctan{x} - x + \arctan{x}  \right) + C
			}

		\qs{Vypočítej integrál}{ $$ \int \frac{7x^2}{\sqrt{1 + x^3}} \diff{x} $$ }
			\note{Substituce}{
				$$ t = 1 + x^3 $$
				$$ \diff{t} = 3x^2 \diff{x} $$
			}
			\mathblock{
				&= \int \frac{7x^2}{\sqrt{t}} \frac{1}{3x^2} \diff{t} \\
				&= \int \frac{7}{\sqrt{t}} \frac{1}{3} \diff{t} \\
				&= \frac{1}{3} \int \frac{7}{\sqrt{t}} \diff{t} \\
				&= \frac{7}{3} \int \frac{1}{\sqrt{t}} \diff{t} \\
				&= \frac{7}{3} \int \frac{1}{t^{\frac{1}{2}}} \diff{t} \\
				&= \frac{7}{3} \int t^{-\frac{1}{2}} \diff{t} \\
				&= \frac{7}{3} \frac{t^{\frac{1}{2}}}{\frac{1}{2}} \\
				&= \frac{7}{3} 2t^{\frac{1}{2}}\\
				&= \frac{14}{3} t^{\frac{1}{2}} + C
			}
		
		\qs{Vypočítej integrál}{ $$ \int \frac{x^9}{\left(1 + x^5\right)^3 } \diff{x} $$ }
			\note{Substituce}{
				$$ t = x^5 + 1 $$
				$$ \diff{t} = 5x^4 \diff{x} $$
			}
			\mathblock{
				&= \int \frac{x^9}{ t^3 } \frac{1}{5x^4} \diff{t} \\
				&= \frac{1}{5} \int \frac{x^9}{ t^3 } \frac{1}{x^4} \diff{t} \\
				&= \frac{1}{5} \int \frac{x^5}{ t^3 } \diff{t} \\
				&= \frac{1}{5} \int \frac{t - 1}{ t^3 } \diff{t} \\
				&= \frac{1}{5} \left( \int t^{-2} \diff{t} - \int t^{-3} \diff{t} \right) \\
				&= \frac{1}{5} \left( -t^{-1} - \frac{t^{-2}}{-2} \right) \\
				&= \frac{1}{5} \left( -t^{-1} + \frac{1}{2}t^{-2} \right) \\
				&= -\frac{1}{5} t^{-1} + \frac{1}{10}t^{-2} + C
			}

		\qs{Vypočítej integrál}{ $$ \int \frac{\diff{x}}{\sqrt{1 - x^2} \cdot \arccos^3{x} } $$ }
			\mathblock{
				\int \frac{1}{\sqrt{1 - x^2}} \cdot \frac{1}{\arccos^3{x}} \diff{x} \\
			}
			\note{Substituce}{
				$$ t = \arccos{x} $$
				$$ \diff{t} = -\frac{1}{\sqrt{1 - x^2}} \diff{x} $$
			}
			\mathblock{
				-\int \frac{1}{t^3} \diff{t} = -\int t^{-3} \diff{t} = \frac{t^{-2}}{2} + C
			}

		\qs{Vypočítej integrál}{ $$ \int \frac{1}{x^2} \sin{\frac{1}{x}} \diff{x} $$ }
			\mathblock{
				\int x^{-2} \sin{x^{-1}} \diff{x} \\
			}
			\note{Substituce}{
				$$ t = x^{-1} $$
				$$ dt = -x^{-2} \diff{x} $$
			}
			\mathblock{
				&= \int \frac{x^{-2}}{1} \sin{(t)} \left( -\frac{1}{x^{-2}} \right)\diff{t} \\
				&= - \int \sin{(t)} \diff{t} \\
				&= - \int \sin{t} \diff{t} \\
				&= - (-\cos{t}) \\
				&= \cos{t} \\
				&= \cos{\frac{1}{x}} + C
			}

		\qs{Vypočítej integrál}{ $$ \int \frac{1}{(1 + x)\sqrt{x}} \diff{x} $$ }
			\note{Substituce}{
				$$ t = \sqrt{x} = x^{\frac{1}{2}} $$
				$$ \diff{t} = \frac{1}{2}x^{-\frac{1}{2}} \diff{x} = \frac{1}{2\sqrt{x}}\diff{x}$$
			}
			\mathblock{
				&= \int \frac{1}{(1 + t^2)t} \cdot \frac{1}{\frac{1}{2\sqrt{x}}} \diff{t}\\
				&= \int \frac{1}{(1 + t^2)t} \cdot \frac{2t}{1} \diff{t} \\
				&= 2 \int \frac{1}{1 + t^2} \diff{t} \\
				&= 2 \arctan{t} \\
				&= 2 \arctan{\frac{1}{x}} + C
			}

		\qs{Vypočítej integrál}{ $$ \int \tan{x} \diff{x} $$ }
			\mathblock{
				\int \frac{\sin{x}}{\cos{x}} \diff{x}
			}
			\note{Substituce}{
				$$ t = \cos{x} $$
				$$ \diff{t} = -\sin{x} \diff{x} $$
			}
			\mathblock{
				&= \int \frac{\sin{x}}{t} \cdot \frac{1}{-\sin{x}}\diff{t} \\
				&= \int \frac{1}{t} \cdot \frac{1}{-1}\diff{t} \\
				&= -\int \frac{1}{t} \diff{t} \\
				&= -\ln{|t|} \\
				&= -\ln{|\cos{x}|} + C
			}
	
\end{document}
